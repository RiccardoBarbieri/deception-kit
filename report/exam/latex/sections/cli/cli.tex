\section{Client linea di comando}

Come indicato in precedenza il metodo principale per interagire con l'API \texttt{deception-core} è una applicazione da linea di comando.

Alla versione corrente l'interfaccia \texttt{deception-cli} supporta il comando \texttt{generate} che accetta i parametri:
\begin{itemize}
    \item \texttt{--component}: del componente da generare
    \item \texttt{--definition}: nome del file che contiene la definizione
\end{itemize}

Un esempio:

\begin{minted}[bgcolor=lightgray,framesep=2mm,baselinestretch=1.2,fontsize=\footnotesize,escapeinside=||,mathescape=true]{bash}
deception-cli generate --component "id-provider" ---definition definition.yaml
\end{minted}

Una volta lanciato il comando l'applicazione si occuperà di:
\begin{itemize}
    \item istanziare il demone \texttt{deception-core} se non ancora in esecuzione
    \item creare l'istanza temporanea di Keycloak
\end{itemize}

Una volta avviato il server Keycloak vengono effettuate le necessarie richieste HTTP al demone.

Una volta registrate tutte le risorse necessarie, il tool sfrutta l'applicativo \texttt{certbot} per generare i certificati TLS, questi verranno inclusi nella definizione dell'immagine.

Completata la generazione delle risorse viene estratta la configurazione Keycloak dall'istanza temporanea e generato il Dockerfile finale con i parametri di interesse correttamente configurati.

Una volta generato il file Dockerfile contenente la definizione dell'immagine viene stampato a video il comando che può essere utilizzato per istanziare un container a partire dalla definizione generata.
