\section{Database}

La prima scelta progettuale effettuata è stata la decisione di quale particolare DBMS utilizzare per la generazione del componente, è stato necessario scegliere un DBMS con distribuzioni sotto forma di immagine OCI e che permettesse di eseguire script di inizializzazione al primo startup per importare i dati generati.

Questa scelta si è successivamente rivelata superflua in quanto tutti i principali DBMS dispongono di definizioni di immagine OCI e supportano inizializzazione tramite script, si è deciso però di supportare PostgreSQL in quanto si tratta della soluzione open source che offre la maggiore quantità di feature avanzate.

Si denota che è comunque possibile istanziare le risorse generate in un server MySQL in quanto i due DBMS condividono tutte le funzionalità di SQL esteso che vengono usate dal generatore così come il meccanismo di inizializzazione che verrà dettagliato in seguito.

\subsection{Definizione componente}
La definizione delle caratteristiche del componente da generare è costituita da una estensione della specifica \underline{deception-def} usata anche per il componente \texttt{idprovider}.

La specifica per il componente \texttt{database} richiede la definizione delle seguenti proprietà:
\begin{itemize}
    \item \texttt{connection}: definisce informazioni per la connesione al server
    \item \texttt{databases}: dichiara i nomi dei database da creare
    \item \texttt{tables}: specifica le caratteristiche delle tabelle da generare (dettagli in seguito)
    \item \texttt{users}: lista di utenti e dichiarazione di un utente admin principale
\end{itemize}

\subsubsection{Tabelle}
Per definire le tabelle da generare e popolare con dati fittizi si è deciso di permettere all'utente di specificare il nome della tabella e il database di appartenenza.

Per creare lo schema per una tabella, l'utente deve definire un \texttt{prompt}: questo testo viene usato dal generatore per generare una serie di nomi ci campi e relativi tipi sfruttando un large language model.

L'utente è inoltre in grado di specificare una serie di metaparametri che impongono al generatore limiti minimi e massimi sul numero di colonne e righe da generare per le tabelle.

\begin{minted}[bgcolor=lightgray,framesep=2mm,baselinestretch=1.2,fontsize=\footnotesize,escapeinside=||,mathescape=true]{yaml}
tables:
  min_columns_per_table: 5
  max_columns_per_table: 10
  min_rows_per_table: 5
  max_rows_per_table: 10
  definitions:
    - name: "flight_schedule"
      prompt: "Flight Schedule"
      database: "flight_db"
    - name: "passengers"
      prompt: "Flight Passengers"
      database: "customer_db"
\end{minted}

\subsubsection{Utenti}
Gli utenti vengono definiti specificando uno username, una password e una serie di database ai quali hanno accesso dove vengono inclusi anche i permessi che possiedono su un determinato database.

\begin{minted}[bgcolor=lightgray,framesep=2mm,baselinestretch=1.2,fontsize=\footnotesize,escapeinside=||,mathescape=true]{yaml}
users:
  definitions:
    - username: user1
      password: password1
      databases:
        - name: flight_db
          permissions:
            - select
            - insert
            - update
\end{minted}

\subsection{Init script}
Le risorse generate a partire dalla \texttt{deception-def} del componente vengono propriamente salvate in appositi file \texttt{.sql}, questi file vengono importati all'interno del container alla sua inizializzazione e successivamente eseguiti dal runtime di PostgreSQL per creare le entità nel server.

Il meccanismo di inizializzazione implementato da PostgreSQL (e MySQL) esegue tutti gli script \texttt{.sql} che trova nella cartella \texttt{/docker-entrypoint-initdb.d/} del container con l'identità dell'utente principale specificato nella definizione del componente. \`E importante specificare che gli script di inizializzazione vengono eseguti solo se non esistono già dati nel database per non entrare in conflitto con dati esistenti.

Questo meccanismo di inizializzione è molto flessibile in quanto permette anche di eseguire script \texttt{bash} complessi per configurare in modo più avanzato il server, questo particolare permetterà in futuro di aggiungere ulteriori configurazioni alla definizione del componente.






