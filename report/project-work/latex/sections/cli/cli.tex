\section{Miglioramenti CLI}

Le modifiche apportate all'interfaccia da linea di comando consistono in:
\begin{itemize}
    \item ping a \texttt{deception-core} per verificare l'esecuzione del server
    \item generazione di componenti multipli
    \item sistema di configurazione
\end{itemize}

\subsection{Generazioni multiple}

La CLI è stata adattata per poter accettare configurazioni multiple con un solo comando.

Data la possibilità di generare più componenti viene aggiunto al nome del Dockerfile il nome del componente generato e sono suddivisi insieme agli asset necessari in apposite cartelle separate.

\begin{minted}[bgcolor=lightgray,framesep=2mm,baselinestretch=1.2,fontsize=\footnotesize,escapeinside=||,mathescape=true]{bash}
deception-cli generate -c "database" -d id.yaml -c "idprovider" -d db.yaml
\end{minted}

\subsection{Configurazione}

\`E stato aggiunto un file di configurazione all'installazione della CLI che permette di specificare proprietà di configurazione del client, al momento accetta soltanto la proprietà \texttt{baseUrl} che indica l'URL di base della API \texttt{deception-core}.

\`E possibile specificare un file di configurazione personalizzato tramite il flag \texttt{-c}.

\begin{minted}[bgcolor=lightgray,framesep=2mm,baselinestretch=1.2,fontsize=\footnotesize,escapeinside=||,mathescape=true]{bash}
deception-cli -c <path-to-config> generate ...
\end{minted}




