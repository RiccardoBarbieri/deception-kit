\section{Introduzione}

In questo documento verrà discusso lo sviluppo di \texttt{deception-kit}, un sistema software in grado di generare servizi e componenti fasulli allo scopo di esporre bersagli a valore nullo a eventuali attori malevoli.

Nel mondo della sicurezza informatica è da tempo diffuso l'utilizzo di misure di difesa basate sull'inganno, queste soluzioni mirano a esporre risorse e servizi che possono essere considerate di particolare interesse per possibili attaccanti; l'approccio principale si basa sulla creazione di un componente \textit{honeypot} come punto di ingresso fasullo al perimetro di sicurezza di una organizzazione.

Un honeypot dovrebbe essere progettato per sembrare ricco di informazioni agli occhi di un potenziale attaccante senza però lasciar trasparire la sua natura di esca esponendo servizi non sufficientemente protetti che possono apparire sospetti ad attori malevoli: l'obiettivo principale è quello di bilanciare questi attributi per indurre un attaccante a \textbf{sprecare} più risorse possibili su un obiettivo futile.

Questo progetto mira a fornire una suite di strumenti atti a semplificare la creazione di suddetti honeypot, offrendo una interfaccia di facile utilizzo che considera tutte le fasi richieste nella generazione di un componente di questo tipo.


