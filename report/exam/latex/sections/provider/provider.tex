\section{Identity provider}

Come primo passo nello sviluppo di questo sistema è stato necessario scegliere l'identity provider sul quale si baserà le generazione e studiarne il funzionamento.

Si è deciso di utilizzare Keycloak, un identity provider open source che offre un varietà molto ampia di configurazioni, adattandosi a svariati casi di utilizzo. Offre tutte le feature che alternative closed source come Okta o Microsoft Azure Active Directory offrono a scapito di una barriera di più alta, ampiamente colmata da una documentazione molto dettagliata e una community altrettanto attiva.

Tuttavia il vantaggio principale che ha guidato questa scelta è la possibilità di effettuare self-hosting del servizio, opzione non disponibile con soluzioni closed source, in particolare Keycloak è predisposto all'utilizzo tramite container Docker, offre anche uno strumento per generare una immagine ottimizzata in base alle configurazioni utilizzate.

\subsection{Concetti chiave}

Le feature che Keycloak offre sono svariate, si elencano di seguito quelle rilevanti ai fini del progetto:
\begin{itemize}
    \item supporto per OpenID Connect
    \item supporto per OAuth2.0
    \item Single Sign On/Out per applicazioni browser
    \item Two/Multi Factor Authentication
\end{itemize}

Keycloak suddivide tutte le configurazioni relative a utenti, client, ruoli, gruppi e flow in realm, ogni realm è isolato dagli altri esistenti e può gestire e autenticare soltanto utenti che controlla per client che controlla.

\subsubsection{Assegnamento ruoli}

I ruoli possono essere assegnati a singoli utenti, client e gruppi.

Ruoli assegnati a utenti possono essere sfruttati dai client per implementare strategie di Role Based Access Control, è possibile anche assegnare ruoli a un particolare gruppo per associare un set di ruoli a un utente.

\subsubsection{Client}

Un client è una entità che può richiedere a Keycloak di autenticare un utente o un altro servizio e ottenere le informazioni di autenticazione relative una volta autenticato.

Keycloak supporta il concetto di \textit{direct grant} permettendo ai client di richiedere accesso ad altri servizi tramite Keycloak, entrano in gioco i ruoli assegnati a singoli client.

